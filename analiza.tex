\section{Analiza}
Aplikacja służąca do wystawiania takich dokumentów jak faktura powinna również dawać możliwość 
pełnego zarządzania właśnie nimi. Urzytkownik oprócz wystawienie powinien móc również dokonac późniejszej
edycji czy usunięcia faktury wystwionej omyłkowo. Aplikacja będzie dawać możliwość wystwiania kilki rodzai
faktur m.in. faktur VAT, faktur VAT marża, korekt, duplikatów dokumentów.
\subsection{Użytkownicy}

\subsection{Aktorzy}
\begin{itemize}
  \item Użytkownik - Ma możliwość wystwiania faktur oraz zarządzania nimi. Aplikacja umożliwia mu również
  zarządzanie płatnościami dkononanymi przez klienta. Będzie mógł równiez dodawać, edytować bądź usuwać 
  klientów z bazy danych. Będzie miał możliwość zarządzania asortymentem dostępnym do sprzedaży. 
  \item Administrator - Oprócz funkcji dostępnych dla użytkownika administrtor może rownież zarządzać
  użytkownikami w systmie.
\end{itemize}
\subsection{Przypadki użycia}
\subsection{Podziała na moduły}
\subsection{Wykorzystanie wybranych technologii w aplikacji}
Aplikacja służąca do wystawiania faktur będzie wykorzystywała klika z dostępnych technologii.
W kolejnych podrozdziałach opiszę w jakim stopniu te technologie będą wykorzystywane.
\subsubsection{Ajax}
Ajax będzie wykorzystywany na trzy różne sposoby:
\begin{itemize}
  \item dynaimiczne tworzenie menu - menu jest tworzone w zależności od dostępnych opcji. W momencie wciśnięcia
  przez użytkownika przycisku menu do serwera wysyłane jest zapytanie o dostępne opcje, a następnie menu 
   jest wyświetlane w oknie przeglądarki.
  
  \item dynamiczne tworzenie listy dostępnych możliwości - użytkownik będzie mógł wyszukiwać znajdujących 
  się w bazie danych elementów takich jak klienci, towary czy usługi za pomocą dynamicznego szukania.
  Po wpisaniu części intesującego użytkownika elementu będą wyświetlane wszystkie dostępne opcje.
  \item Zmiany na stronie nie wymagające przeładowania całej strony - Po wybraniu przez użytkownika interesującego
   elementu nie będzie pobierana z serwera cała strona, a jedynie odpowiedź o powodzeniu wykonania operacji,
   tym samym będzie wprowadzona modyfikacja niezbędnych elemenentow na stronie.
\end{itemize}
\subsubsection{Dojo}
Biblioteka ta posiada zbiór wielu ciekawych kontrolek oraz umożliwia wprowadzanie zmian na stronie poprzez 
dostęp do elementów DOM strony. W swojej aplikacji z pewnością urzyję tych możliwości a także menu kontekstowe
i wyskakujące okienka będą wykonane z pomocą tej biblioteki.
\subsubsection{ObjectLedge}
Szkielet aplikacyjny który umożliwi stworzenie alikacji od podstaw zostaną wykorzysten możliwości dotyczące 
modułu security, w ten sposób będzie zrealizowana autoryzacja oraz możliwości zarzadzania użytkownikami w 
systemie.
\subsubsection{Hibernate}
Będzie wykorzystana do mapowania obiektowo relacyjnego, dzięki temu w aplikacji nie będzie konieczności 
odwoływania się do bezpośrednio do bazy danych w sposób relacyjny, a odwołana do bazy danych będą bardziej 
obiektowe niż miałoby to miejsce w przypadku bezpośredniego odwolania od bazy danych.
\subsubsection{JQuery}
Dzięki tej bibilotece w aplikacji będzie można umieścić kilka ciekawych kontrolek dzięki czemu 
aplikacja będzie wyglądać atrakcyjniej i mam nadzieje będzie bardzie przyjazna dla użytkownika.

