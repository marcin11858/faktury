

\section{Faktury}
\subsection{Wstęp}
Wraz z rozwojem cywilizacji pojawiła się potrzeba dokumentowania zawieranych
transakcji. Sytuacja ta wymusiła powstanie spójnego dokmentu, który będzie
zawierał wszystkie najważniejsze informacje dotyczące transakcji, a także będzie
miał spójny wygląd. Obecnie w Polsce o wyglądzie, sposobie przechowywania tych
dokumentów decyduje Rozporządzenie Ministra
Finansów, którego treść możemy odnaleźć tutaj\cite{RozporzadzenieFaktury}.

Z punktu widzenia państaw faktura stanowi ważny dokument księgowy, który może
pełnić wiele istotnych funkcji, takich jak: obliczenie podatku,
wyznaczenie obrotu przedsiębiorcy. W Polsce wykorzystuje się również ten
dokument jako podstawę do uzyskania ulg czy zwolnień.



Faktura jest to dokument sprzedaży. Dokument ten zawiera szczegółowe informacje
o tym jaki towar został sprzedany, bądź jaka usługa została wykonana. Każda
pozycja faktury zawiera szczegółowe informacje o tym: w jakiej
ilości, po jakiej cenie oraz jaki podatek został nałożony na ten towar bądź
usługę.
Dokument ten również jednoznacznie identyfikuje strony biorące udział w
transkacji, opisuje również atakie aspekty jak sposób płatności czy datę
doknania transkazji.

Faktura jest dokumentem, który możemy spotkać w niemal każdym przedsiębiorstwie.
Począwszy od małych sklepików, poprzez stacje benzynowe czy hotele a kończąc na dużych
hurtowniach, w każdej z dziedzin życia wykorzystywane sa faktury.

Faktura jest specyficznym dokumentem ponieważ sposób
przechowywania oraz księgowania określa państwo. Treść tego dokumentu nie
może się zmieniać wraz z upływem lat, raz wystawiona faktura nie może być
później zmieniona przez którąkolwiek ze stron. 

\subsection{Klienci}
\label{klienci}
Nie ważne od rodzaju prowadzonej działalności dla przediębiorcy najważniejsi są
klienci, którzy pozwalają mu przeżyc na wolnym rynku. Każde z przedsiębiorstw
posiada pewną baze danch klientów, którym wystawia faktury. Sposób
przechowywania takiej bazy może być różny, zależnie od charakteru jak i od jej
rozmiarów.Woubraźmy sobie małą księgarnie gdzie faktury wystawiane są ręcznie, a
ilość klientów jest niewielka. Często mamy tam małą baze spisaną przez miłą
panią w małym zeszyciku, który to zeszyt pełni rolę swoistej tabeli bazy danych. Umożliwia to tej pani szybsze znalezienie wszystkich informacji o kontrahecie, którego dane już widniają w zeszycie. Kożystanie z takiego zeszytu może przyspieszyć momen
wystawiania faktury pownieważ wspomniana pani nie musi za każdym razem prosić o
pełne dane nabywcy. Zeszyt ten nie spawdzi sie natomiast w firmie gdzie
liczba klientów jest duża. W dużych przedsiębiorstwach baza danych klientów
klientów jest już przechowywana w formie elektronicznej, która przyspiesza
proces wyszukiwania oraz pielengnowania danych klientów. Mali przedsiębiorcy
równeż stosują tekie rozwiązania. Niezależnie od sposobu przechowywania klient
jest dodawany do bazy danych najpóźniej w momencie wystawienia pierwszej faktury
dla niego. 

Dzięki bazie danych klientów możemy każdego z nich traktować indywidualnie, ma
to ogromne znaczenie gdy chcemy zachęcać do dalszych zakupów w naszej placówce
poprzez udzielanie rabatów. Możemy dzięki temu przypisać większy rabat klientom,
którzy kupują o nas więcej. Skoro już wspomnieliśmy o kwestii finansowej to taka
baza danych umożliwia nam także konrtole płatości dokonywanych przez klienta i
odpowiednia reakcję w zależności od tego czy klient wywiązuje się z ciążacych na
nim obciążeń czy też nie.
\subsection{Towary oraz usługi}
Przedsiębiorstwa możemy podzielić na dwa główne rodzaje czyli te, które świadczą
usługi oraz te które dokonują sprzedaży towarów. Oczywiście istnieją firmy,
które łączą oba te rodzaje. Natmosias chcialem w ten sposób pokazać że
niezależnie od tego czy firma zajmuje się świdczeniem usług czy też obrotem
towarem wymaga by za określone dobra trzymać  Handel opiera się przedewszystkim
sprzedaży towarów i/lub świadczeniu usług bez tego przedsiębiorstwo z całą pewnością nie może istnieć.   Głównym elemetnem handlu są towary i/lub usługi oferowane przez przedsiębiorstwo.

W gospodarce wolnorynkowej przedsiębiorstwo wykonuje usługi bądź sprzedaje
towary i za nie orzymuje pieniądze. Jest to najważniejszy element
przedsiębiorstwa, ponieważ przedsiębiorstwa które nie mają w swojej ofercie
żadnych towarów i/lub usług praktycznie nie istnieją.

Zależnie od profilu przedsiębiorstwa możemy wyróżnić takie w kótrych liczba
towarów i / lub usług jest niewielka oraz takie w kótrych ta liczba jest
znacząca. Do pierwszego rodzaju należa np. stacje benzynowe gdzie najmniejsze
posiadają w swojej ofercie kilka, no może klikanaście produktów a na fakturach
takiej stacji najczęściej widnieje jeden z rodzai paliw płynnych. W przypadku
tak małej ilości oferowanych przez przedsiębiorstwo dóbry nie musimy się
specjalnie zajmować koniecznością oznaczenia symbolem towaru w celu szybszego
odnalezienia. Zupełnie inną rolę pełni symbol towaru w dużej hurtowni, która
posiada w swojej ofercie klika czy klikanaście tysięcy towarów. W takim
przypadku konieczne jest właściwe oznaczenie towarów za pomocą wzdefiniowanych
przez siebie symboli pełniących rolę indywidualnego id każdego towaru. Można
używać naturalnie używac już istniejących identyfikatorów takich jak kody EAN
widniejące na towarach, jednak takie roziwązanie jest dość niepraktyczne
ponieważ za każdym razem musielibyśmy wpisywać ciąg trzynastu liczb, co jest
czasochłonne. Ponieważ przeciętny człowniek nie jest w stanie zapamiętać
klikunastu tysięcy kodów, oprócz szukania po indywidualnych numerach id towarów
musimy zadbać o możliwość wyszukiwania towarów po nazwie.
\subsection{Rodzaje faktur}
\label{rodzajeFaktur}
Ponieważ istnieją różne rodzaje transkacji, a także różne podmioty mogą być
stronami transakcji. Pojawiła się konieczność opracowania odmiennych rodzai
faktur do różnych zastosowań. Mnogość dostępnych form faktur obecnie zaspokaja
potrzeby kupców oraz sprzedawców. W kolejnych podrozdziałach postaram się
pokrótce opisać każdy z istniejących rodzai. Każdy z rodzai rozpoznawany jest
przez charakterystyczny dla siebie nagłówek.
\subsubsection{Faktura VAT}
Najcześciej spotykana forma faktury, przykładem tego typu dokumentu jest
rachunek za telefon, prąd czy gaz.
W przypadku takich rachunków sprzedacą jest przedsiębiorstwo natomiast nabywca to najczęściej osoba fizyczna.
Fakturę tego rodzaju można poznać po nagłówku FAKTURA VAT.

Do wystawienia faktury VAT mają prawo podatnicy będący czynnymi podatnikami VAT,
muszą oni również posiadać numer identyfikacji podatkowej. W przypadku drugiej
strony transakcji nie jest ona zobligowania do posiadania numeru identyfikacji
podatkowej. W skutek czego tego typu faktura może być wystawiona również osobie
fizycznej.


\subsubsection{Faktura VAT marża}
Tego typu faktura wystawiana jest w przypadku gdy podstawą opodatkownia jest
marża, czyli różnica między kwotą zapłaconą przez kupującego a ceną nabycia
przez sprzedającego.


Tego typu dokument wystwiany jest w przypadku: świadczeniu usług
turystycznych, dostawie towarów używanych, dzieł sztuki, przedmiotów
kolekcjonerskich a także antyków. Cechą charakterstyczną tego rodzaju faktury
jest, jak ma to miejsce w przypadku pozostałych faktur nagłówek, który w tym
przypadku brzmi: "Faktura VAT marża".
\subsubsection{Faktura VAT-MP}
Ten rodzaj faktur wystawiany jest przez małych podatników. Aby przedsiębiorca
mógł być uznawany za małego musi spełniać wymagania okerślone w Ustwie o
podatku od towarów i usług. W świetle obowiązującyhc przepisów małym
podatnikiem jest podmiot, którego wartość sprzedaży razem z podatkiem nie
przekroczyła 1,2 mln Euro (zgodnie z średnim kursem Euro na pierwszy dzień
roboczy października poprzedniego roku). Tego typu podatnik to również
przedsiębiorstwo maklerskie kótrego dochód wynikajacy z świadczonych usług nie
przekryczył 45 000 Euro. Tego typu faktury są bardzo rzadko spotykane. Ten
rodzaj faktury można poznać po nagłówku "Faktura VAT-MP".
\subsubsection{Faktura Korygująca}
Ponieważ jak w każdej dziedzinie życia tak i w przypadku wystwaniania faktur
istnieje ryzyko wystąpienia jakiejś pomyłki, ustawodawca przewidział możliwość
skorygowanie zaistniałego błędu za pomocą faktury korugującej zwanej też Korektą
faktury.


Pomyłka ta może dotyczyć ceny, stawki podatku VAT, ilości a także przy innych pomyłkach dotyczących
pozycji wyszczególnionych na fakturze. Na tym dokumenice muszą być wymienione
pomyłkowe wartości pozycji faktury jaki i również pozycje poprawnine. Tego
rodzaju fakturę wystawia się również w przypadku gdy srzedawca udziela nabywcy rabatu po wystawieniu faktury. Z racji tego że faktura korygująca dotyczy innej
faktury uprzdednio wsytawionej konieczne jest umieszczenie na niej numeru
faktury, której dotyczy korekta. Przedsiębiorcy zobowiązani są do przechowywania
zarówno błędnej faktury jak i faktury korygującej. Dokument tego typo powinien w
nagłówku zawierać wyraz "KOREKTA" albo wyrazy "FAKTURA KORYGUJĄCA".

\subsubsection{Nota korygująca}
Jest to specyficzny rodzaj dokumentu, który w odróznieniu od innych rodzai
faktur jest wystawiany przez nabywcę w wyniku zauważnie błędów w fakturze bądź
korekcie faktury. Błędy te mogą dotyczyć: nazw stron transakcji oraz nazw
towarów bądź usług zawartch w wystawionym wcześniej dokumencie. Można rozpoznać
ten typ dukumentu po wyrazach "NOTA KORYGUJĄCA".

Nabywca po wystawieniu noty korygującej przesyła orginał i kopie sprzedawcy i
ten w zależności od tego czy się zgadza z jej treścią czy też nie, potwierdza
ten fakt podpisem bądź nie.
\subsubsection{Faktura pro-forma}
Ten typ faktury w przeciwienstwie do innych faktur nie stanowi dowodu
księgowego, co wiąże się z tym że żadna ze stron nie ma obowiązku księgowania
faktur pro-forma. Przepisy nie wspominają nic o tego typu dokumencie natomiast
pewnym jest że nie może być on wsytawiony po otrzymaniu zaliczki albo wydaniu
towaru bądź wykonaiu usługi. Wynika z tego że taka faktura może być uznana jako
potwierdzenie zawarcia transakcji. Ustawodawca nie zdefiniował nagłówka tego
dokumentu lecz w praktyce umiesza się w nim wyrazy "FAKTURA PROFORMA".
\subsubsection{Faktura zaliczkowa}
Taką fakturę
wystwia się przed wydaniem towaru bądź wykonaniem usługi.Tego rodzaju faktura nie różni się znacząco od Faktury VAT czy Faktury VAT-MP
istotną cechą jest moment wystawienia tej fakrtry, który wynosi 7 dni od
otrzymania zaliczki za wyszczególnione na niej towary bądź usługi. Zawiera ona
również datę oraz kwotę zaliczki.
\subsubsection{Dupikat faktury}
Wystawiany jest na wniosek nabywcy przez sprzedawce w momencie gdy orginał
faktury bądź korekty ulegnie zniszczeniu lub zagninięciu. Duplikat, jest
wystawiany na podstawie kopii orginalnej faktury, zawiera wszystkie te
informacje które zawierał orginał faktury, a dodatkowo powinien zawierać słowo "DUPLIKAT" oraz date jego wystawienia. Tak samo jak w przypadku
innych rodzai faktur dupilikat rownież wystawiany jest w dwóch elgemplarzach z
czego orginał otrzymuje nabywca.
\subsection{Podstawowe elementy faktury}
Faktura jest dokumentem którego wygląd ściśle określają przepisy prawa.
W przepisach jest jasno określone co powinna zawierać faktura aby była poprawna.
Nieprzestrzeganie tych zaleceń może powodowac poważne konsekwencje prawne
zarówno dla przedsiębiorcy wystawiającego fakturę jaki i dla twórcy
oprogramowania, które towrzy niepoprawne faktury. Ponieważ tak jak w każej
dzienie życia prawo ciągle ewoluuje tak i tu następują ciągłe zmiany ktore
stwarzają konieczność aktualizacji na bierząco stworzonego oprogramowania. Niektóre zmiany
można przewidzieć jak np. zmiany stawek podatku VAT, które nastąpiły na
przełomie 2010 i 2011 roku, jest niemal pewne że kolejna zmiana stawek tego
podatku nastąpi pod koniec 2013 roku.
Niestety nie wszystkie zmiany da się przewidzieć, przykładem tego typu
nowielizacji może być ustawa, która wymusiła na twórcach oprogramowania
aktualizacje jest zastąpnienie w 2002 roku symbolu SWW (Systematyczny Wykaz
Wyrobów) symbolem PKWiU (Polska Klasyfikacja Wyrobów i Usług).
\subsubsection{Nagłówek}
Pozwala nam na pierwszy rzut oka okreslić z jakiego typu fakturą mamy do
czynienia.. Typy faktur zostały
określone w rozdziale \ref{rodzajeFaktur}. 
\subsubsection{Numer porządkowy}
Kolejny mumer faktury oznaczonej wybranym nagłowkiem. Ważne jest aby nie
istniały każda z faktur określnego typu miała unikalny numer, a także aby
numeracja była ciągła. Błędy w numeracji stwarzają podejrzenie popełnienia
przestępstwa przez przedsiębiorcę.
class Klasa {
}
Istnieją różne rodzaje numeracji, które są wykorzystywane zależnie od potrzeb i
wielkości przedsiębiorstwa wystwiającego fakturę. Najpopularniejszym sposobem
podanie roku oraz numeru porządkowego faktury w tym roku. W przypadku firm,
które wystawiają duże ilości faktur stosuje się również numer miesiąca.
Natomiast jeśli przedsiębiorstwo posiada kilka filii umieszczany jest również
numer filii przedsiębiorstwa.
\subsubsection{Nazwy obu stron transakcji}
W przypadku przedsiębiorstw faktura zawierać nazwy obu
stron biorących udział w transkacji, mogą być to również skrócone nazwy. Jeśli
którąś ze stron nie jest przedsiębiorstwo powinny byc umieszczone tam imiona i
nazwiska nabywcy i sprzedawcy. Konieczne jest równiez umieszczenie adresów
nabywcy oraz sprzedawcy.
\subsubsection{Numer identyfikacyjny podatnika}
Na fakturze musi być umieszczony numer indentyfikacyjny sprzedawcy, natomiast w
przypadku nabywcy nie jest to konieczne jeśli nabywca jest osobą fizyczną.
\subsubsection{Data}
Faktura musi zawierać dokładną datę wystawienia. Data sprzedaży nie jest
konieczna jeśli data sprzedaży jest inna od daty wystawienia faktury. Data
sprzedaży może zostać uproszczona do podania tylko miesiąca i roku w przypadku
sprzedaży o charakterze ciągłym.
\subsubsection{Nazwa towaru bądź usługi}
Nazwa powinna jednoznacznie definiować sprzedany towar. Powinna precyzyjnie
określać sprzedawany towar bądź wykonywaną usługę. W praktyce nie używa się
długich, opisowych nazw lecz raczej kilku słów określojących: typ produktu,
producenta i rodzaj. Niektóre z firm w nazwach podają rownież numery indeksu lub
inne symbole które są krótsze od popularnych kodów EAN a umożliwiają wyszukanie
towaru.
\subsubsection{Miara i ilość}
Na fakturze musi być wyszczególniona ilość sprzedanych towarów bądź zakres
wykonanych usług. Miara czyli sposób w jaki określana jest ilość towarów bądź
zakeres wykonywanych usług, najszęściej spotykane miary to: sztuka, kilogram,
metr kwadratowy itp. W przypadku towarów których naturalną jednstką miary jest
waga podaje się ją z dokładnością do gramów.
\subsubsection{Cena jednostkowa netto}
Cena jednostkowa towaru bądź usugi bez podatku VAT.
\subsubsection{Wartość netto}
Wartość towarów bądź usług bez podatku VAT czyli ilość pomnożona przez cene
jednostkową. Kwoty na fakturze zaokrągla się do pełnych groszy a ustawodawca
określił że końcówki poniżej 0,5 gr pomija się natomiast te o wartości 0,5 gr i
wyższe zaokrągla się do pełnych groszy.
\subsubsection{Stawki podatku VAT}
Określają jaki podatek jest nałożony na wybrany towar bądź usługę, szczegółowe
informacje o kwotach podatku nałożonych na wybrane towary i usługi można
zlaleźć w ustawie o podatku od towarów i usług\cite{etykieta1}. Obecnie
obowiązujące stawki podatku VAT to:
\begin{itemize}
  \item 23\% podstawowa stawka podatku VAT
  \item 8\%
  \item 5\%
  \item 0\%
  \item zwolniony
\end{itemize}
Stawki podatku VAT podlegają jednak zmianom, obecne stawki podatku obowiązują do
końca 2013 roku.
\subsubsection{Sumy wartość sprzedaży netto}
Suma sprzedanych towarów lub wykonanych usług bez podatku VAT wraz z podziałem
na stawki podatku.
\subsubsection{Kwoty podatku}
Kwoty podatku z podziałem na poszczególne stawki. W przypadku gdy przedsiębiorca
wystawiając fakturę podaje ceny brutto towarów bądź usług kwotę podatku można
obliczyć z następującego wzoru:
\begin{equation}
KP = \frac{WB * SP}{100 + SP}
\end{equation}
gdzie:

KP - kwota podatku z podziałem na poszczególne stawki.

WB - zsumowana wartość sprzedaży brutto

SP - stawka podatku.\newline


Innego wzoru używa się również w przypadku faktury zaliczkowej:
\begin{equation}
KP = \frac{ZB*SP}{100+SP}
\end{equation}
gdzie:

KP - kwota podatku z podziałem na poszczególne stawki

ZB - kwota otrzymanej całości lub części należności brutto

SP - stawka podatku VAT

\subsubsection{Suma}
Kwota należna za wykonaną usługę lub kupiony towar wraz z podatkiem VAT. Na
fakturze umieszcza się również kwotę zapisaną słownie, ma to znaczenie raczej
historyczne, ponieważ kiedyś faktury w większości były dokumentem wystawianym
ręcznie i można było w ten sposób sprawdzić czy sprzedawca poprawnie wpisał
sumę.
\subsubsection{Sposób płatności}
Określa formę opłaty za fakturę, ponieważ nie zawsze faktura jest opłacana
natychmiast po wystawieniu istnieją różne sposoby jej uregulowania. Część firm
wystawia dowód wpłaty za kilka faktur np. z całego tygodnia czy miesiąca, inne
natomiast wymagają natychmiastowej płatności, a jeszce inne dają kilka dni na
uregulowanie należności.
\subsection{Opis pozycji faktury}

\subsection{Dodatkowe elementy faktury}
\subsubsection{Numer rejstracyjny pojazdu}
W przypadku sprzedaży paliw płynnych konieczne jest umieszczenie numeru
rejstracyjnego pojazdu, który został zatankowany. Często twórcy aplikacji
zapominają o tej konieczności a przedsiębiorcy umieszczają numer rejstracyjny w
uwagach do faktury, co jest zgodne z prawem ponieważ ustawodawca nie definiuje
gdzie powinien się ten numer znajdować.
\subsubsection{Cena towaru brutto}
Nie jest wymaganym elementem faktury ale jest dość użyteczna ponieważ nabywca
bez zbędnego liczenia widzi ile zapłacił za wybrany towar albo usługę.
\subsubsection{Detaliczna cena towaru}
Wiele przedsiębiorstw umożliwia swoim klientom ustawienie marży na zakupiony
towar i umieszczenie na fakturze kolumny zawierającej cene towaru po dodaniu
marży nabywcy. Taka kolumna jest przydatna np. gdy nabywca wprowadza do swojego
systemu magzaynowego zakupione towary.
\subsubsection{Rabat}
\subsubsection{Numer konta}
\subsubsection{Podpis sprzedawcy i nabywcy}
\subsection{Wydruk}
\subsection{Płatności}
\section{Zakres pracy}
W swojej aplikacji skupie się na najpopularniejszych i najczęściej
wykorzystywanych typach faktur. Zaimplementowane typy pozwolą na sprawne i pełne
prowadenie przedsiębiorstwa. Aplikacja będzie umożliwiała wystawianie faktur
VAT, korygowanie tych faktur, a także wystwianie dupikatów do już istniejących w
programie dokumentów. Będzie również możliwość wystawiania faktur Proforma. 

Wszystkie wystawione dokumenty będą przechowywane w historii dzięki czemu będzie
możliwość ich edycji oraz usunięcia w razie pomyłki. Dokumenty będą
przechowywane w kolejności chronlologicznej względem daty wystawienia. Dla
poprawnienia przejrzystości prezentowanego archiwum będzie można ograniczyć
wyśiwetlanie do określonych typów dokumentów.

Dla każdego przedsiębiorcy najważniejszy jest obrót kapitałem oraz panowanie nad
kapitałem w związku z tym aplikacja będzie umożliwiała zarządzanie płatnościami
za wystawione faktury. Użytkownik będzie mógł przeglądać historie płatności
kontahentów, sprawdzać saldo niezapłaconych faktur czy też wyświetlić wartość
dokumentów które zostały uregulowane w danym dniu, tygodniu czy miesiącu.

Aplikacja oprócz przechowywania historii faktur będzie możliwość przechowywania
bazy danych klientów oraz oferowanych towarów i usług. Aby przyspieszyć
tworzenie faktur, klientów będzie można wyszukiwać po nazwie, symbolu
numerycznym lub po numerze NIP natomiast towary i usługi po zapisanej w bazie
danych nazwie.

\section{Wymagania}
\subsection{Wymagania funkcjonalne}
Wymagania funkcjonalne aplikacji:
\begin{itemize}
  \item tworzenie, edycja i usuwanie faktur VAT
  \item korekta już wystawionych faktur VAT
  \item wystawianie dupilikatów zarówno faktur VAT jak i korekt
  \item dodawanie, edycja i usuwanie towarów bądź usuług
  \item dodawanie, edycja i usuwanie kontrahentów
  \item drukowanie dokumentów
  \item zarządzanie płatnościami
  \item 
  \item możliwość dostępu przez przeglądarkę
  \item 
\end{itemize}
\subsection{Wymagania niefunkcjonalne}
Wymagania niefunkcjonalne aplikacji:
\begin{itemize}
  \item aplikacja interaktywna
  \item ilość przechowywanych kontrachentów min. 200
  \item ilość przechowywanych towarów i usług min. 2000
  \item sumaryczna ilość przechowywanych faktur min. 1000
  \item system opercyjny Linuks lub Windows
  \item 
\end{itemize}

\section{Struktura aplikacji}
Jak każda rozbudowana aplikacja tak również moja aplikacja będzie podzielona na
współpracujące ze sobą moduły. Ułatwi to późniejszą pracę nad doskonaleniem i
pielęgnacją mojej aplikacji.
\subsection{Moduł zarządzania fakturami}
\subsection{Moduł zarządzania klientami}
\subsection{Moduł zarządzania towarami i usługami}
\subsection{Moduł zarządzania płatnościami}

\newpage
\begin{thebibliography}{szerokosc}
\bibitem{etykieta1}Obwieszczenie Marszałka Sejmu Rzeczypospolitej Polskiej z
dnia 29 lipca 2011 r. w sprawie ogłoszenia jednolitego tekstu ustawy o podatku
od towarów i usług. (Dz.U. 2011 nr 177 poz. 1054) 
\bibitem{RozporzadzenieFaktury}Rozporządzenie Ministra Finansów z dnia 28 marca
2011 r.
w sprawie zwrotu podatku niektórym podatnikom, wystawiania faktur, sposobu ich
przechowywania oraz listy towarów i usług, do których nie mają zastosowania
zwolnienia od podatku od towarów i usług. (Dz.U. 2011 nr 68 poz. 360)
\end{thebibliography}

